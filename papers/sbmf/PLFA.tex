%\documentclass[runningheads,oribibl]{llncs}
\documentclass[runningheads]{llncs}
\usepackage{natbib}
% \usepackage{stmaryrd}
% \usepackage{proof}
% \usepackage{amsmath}
% \usepackage{amssymb}
% \usepackage{color}
% \usepackage{bbm}
% \usepackage[greek,english]{babel}
% \usepackage{ucs}
% \usepackage[utf8x]{inputenc}
% \usepackage{autofe}
\usepackage{agda}
\usepackage{revsymb}
\usepackage{semantic}
\usepackage{graphicx}
\usepackage{url}
\renewcommand\UrlFont{\color{blue}\rmfamily}
\usepackage{stix}
\setcitestyle{round,aysep={}}

\begin{document}
\title{Programming Language Foundations in Agda}
\author{Philip Wadler}
\authorrunning{P. Wadler}
\institute{University of Edinburgh\\
  \email{wadler@inf.ed.ac.uk}}
\maketitle
%
\begin{abstract}
One of the leading textbooks for formal methods is
\emph{Software Foundations} (SF), written by Benjamin Pierce in
collaboration with others, and based on Coq. After five years using SF
in the classroom, I have come to the conclusion that Coq is not the
best vehicle for this purpose, as too much of the course needs to
focus on learning tactics for proof derivation, to the cost of
learning programming language theory.  Accordingly, I have written a
new textbook, \emph{Programming Language Foundations in Agda} (PLFA).
PLFA covers much of the same ground as SF, although it is not a
slavish imitation.

What did I learn from writing PLFA? First, that it is possible. One
might expect that without proof tactics that the proofs become too
long, but in fact proofs in PLFA are about the same length as those in
SF. Proofs in Coq require an interactive environment to be understood,
while proofs in Agda can be read on the page.  Second, that
constructive proofs of preservation and progress give immediate rise
to a prototype evaluator. This fact is obvious in retrospect but it is
not exploited in SF (which instead provides a separate normalise
tactic) nor can I find it in the literature.  Third, that using raw
terms with a separate typing relation is far less perspicuous than
using inherently-typed terms. SF uses the former presentation, while
PLFA presents both; the former uses about 1.6 as many lines of Agda
code as the latter, roughly the golden ratio.

The textbook is written as a literate Agda script, and can be found here:
\begin{center}
\url{https://plfa.inf.ed.ac.uk}
\end{center}
\keywords{Agda \and Coq \and lambda calculus \and dependent types.}
\end{abstract}

\section{Introduction}

The most profound connection between logic and computation is a pun.
The doctrine of Propositions as Types asserts that a certain kind of formal
structure may be read in two ways: either as a proposition in logic or
as a type in computing.  Further, a related structure may be read as
either the proof of the proposition or as a programme of the
corresponding type.  Further still, simplification of proofs
corresponds to evaluation of programs.

Accordingly, the title of this paper, and the corresponding textbook,
\emph{Programming Language Foundations in Agda} (hence, PLFA)
also has two readings.  It may be parsed as ``(Programming Language)
Foundations in Agda'' or ``Programming (Language Foundations) in
Agda''---specifications in the proof assistant Agda both describe
programming languages and are themselves programmes.

Since 2013, I have taught a course on Types and Semantics for
Programming Languages to fourth-year undergraduates and masters
students at the University of Edinburgh.  An earlier version of that
course was based on \emph{Types and Programming Languages} by
\citet{Pierce-2002}, but my version was taught from its successor,
\emph{Software Foundations} (hence, SF) by \citet{Pierce-et-al-2010},
which is based on the proof assistance Coq \citep{Huet-et-al-1997}.
I am convinced by the claim of \citet{Pierce-2009}, made in his ICFP
Keynote \emph{Lambda, The Ultimate TA}, that basing a course around a
proof assistant aids learning.

However, after five years of experience, I have come to the conclusion
that Coq is not the best vehicle.  Too much of the course needs to
focus on learning tactics for proof derivation, to the cost of
learning the fundamentals of programming language theory.  Every
concept has to be learned twice: e.g., both the product data type, and
the corresponding tactics for introduction and elimination of
conjunctions.  The rules Coq applies to generate induction hypotheses
can sometimes seem mysterious.  While the \texttt{notation} construct
permits pleasingly flexible syntax, it can be confusing that the same
concept must always be given two names, e.g., both
\texttt{subst~N~x~M} and \texttt{N~[x~:=~M]}.  Names of tactics are
sometimes short and sometimes long; naming conventions in the standard
library can be wildly inconsistent.  \emph{Propositions as types} as a
foundation of proof is present but hidden.

I found myself keen to recast the course in Agda \citep{Bove-et-al-2009}.
In Agda, there is
no longer any need to learn about tactics: there is just
dependently-typed programming, plain and simple. Introduction is
always by a constructor, elimination is always by pattern
matching. Induction is no longer a mysterious separate concept, but
corresponds to the familiar notion of recursion. Mixfix syntax is
flexible while using just one name for each concept, e.g.,
substitution is \texttt{\_[\_:=\_]}. The standard library is not perfect, but
there is a fair attempt at consistency. \emph{Propositions as types} as a
foundation of proof is on proud display.

Alas, there is no textbook for programming language theory in
Agda.  \emph{Verified Functional Programming in Agda} by \citep{Stump-2016}
covers related ground, but focusses more on programming with dependent
types than on the theory of programming languages.

The original goal was to simply adapt \emph{Software Foundations},
maintaining the same text but transposing the code from Coq to Agda.
But it quickly became clear to me that after five years in the
classroom I had my own ideas about how to present the material.  They
say you should never write a book unless you cannot \emph{not} write the
book, and I soon found that this was a book I could not not write.

I am fortunate that my student, Wen Kokke, was keen to help.  She
guided me as a newbie to Agda and provided an infrastructure that is
easy to use and produces pages that are a pleasure to view.
The bulk of the book was written January--June 2018, while on
sabbatical in Rio de Janeiro.

This paper is a personal reflection, summarising what I learned in the
course of writing the textbook. Some of it reiterates advice that is
well-known to some members of the dependently-typed programming
community, but which deserves to be better known.  The paper is
organised as follows.

Section~2 outlines the
topics covered in PLFA, and notes what is omitted.

Section~3 compares Agda and Coq as vehicles for pedagogy.  Before
writing the book, it was not obvious that it was even possible;
conceivably, without tactics some of the proofs might
balloon in size. In fact, it turns out that for the results in PLFA
and SF, the proofs are of roughly comparable size, and (in my opinion)
the proofs in PLFA are more readable and have a pleasing visual
structure.

Section~4 observes that constructive proofs of progress and
preservation combine trivially to produce a constructive evaluator for
terms.  This idea is obvious once you have seen it, yet I cannot
find it described in the literature.  For instance, SF separately
implements a \texttt{normalise} tactic that has nothing to do with
progress and preservation.

Section~5 claims that raw terms should be avoided in favour of
inherently-typed terms.  PLFA develops lambda calculus with both raw
and inherently-typed terms, permitting a comparison.  It turns out the
former is less powerful---it supports substitution only for closed
terms---but significantly longer---about 1.6 times as many lines of code,
roughly the golden ratio.

I will argue that Agda has advantages over Coq for
pedagogic purposes.  My focus is purely on the case of a proof assistant
as an aid to \emph{learning} formal semantics using examples of
\emph{modest} size.  I admit up front that
there are many tasks for which Coq is better suited than Agda.
A proof assistant that supports tactics, such as Coq or Isabelle,
is essential for formalising serious mathematics,
such as the Four-Colour Theorem \citep{Gonthier-2008},
the Odd-Order Theorem \citep{Gonthier-et-al-2013},
or Kepler's Conjecture \citep{Hales-et-al-2017},
or for establishing correctness of software at scale,
as with the CompCert compiler \citep{Leroy-2009,Kastner-et-al-2017}
or the SEL4 operating system \citep{Klein-2009,O'Connor-2016}.

\section{Scope}

PLFA is aimed at students in the last year of an undergraduate
honours programme or the first year of a master or doctorate degree.
It aims to teach the fundamentals of operational semantics of
programming languages, with simply-typed lambda calculus as the
central example.  The textbook is written as a literate script in Agda.
As with SF, the hope is that using
a proof assistant will make the development more concrete
and accessible to students, and give them rapid feedback to find
and correct misaprehensions.

The book is broken into two parts. The first part, Logical Foundations,
develops the needed formalisms.  The second part, Programming Language
Foundations, introduces basic methods of operational semantics.
(SF is divided into books, the first two of which have the same names
as the two parts of PLFA, and cover similar material.)

Each chapter has both a one-word name and a title, the one-word name
being both its module name and its file name.

\subsection*{Part I, Logical Foundations}

\paragraph{Naturals: Natural numbers}
Introduces the inductive definition of natural numbers in terms of
zero and successor, and recursive definitions of addition,
multiplication, and monus. Emphasis is put on how a tiny description
can specify an infinite domain.

\paragraph{Induction: Proof by induction}
Introduces induction to prove properties
such as associativity and commutativity of addition.
Also introduces dependent functions to express universal quantification.
Emphasis is put on the correspondence between induction and recursion.

\paragraph{Relations: Inductive definitions of relations}
Introduces inductive definitions of less than or equal on natural numbers,
and odd and even natural numbers.
Proves properties such as reflexivity, transitivity, and
anti-symmetry, and that the sum of two odd numbers is even.
Emphasis is put on proof by induction over evidence that a relation holds.

\paragraph{Equality: Equality and equational reasoning}
Gives Martin L\"of's and Leibniz's definitions of equality, and proves
them equivalent, and defines the notation for equational reasoning used
throughout the book.

\paragraph{Isomorphism: Isomorphism and embedding}
Introduces isomorphism, which plays an important role in the
subsequent development.  Also introduces dependent records, lambda
terms, and extensionality.

\paragraph{Connectives: Conjunction, disjunction, and implication}
Introduces product, sum, unit, empty, and function types, and their
interpretations as connectives of logic under Propositions as Types.
Emphasis is put on the analogy between these types and product, sum,
unit, zero, and exponential on naturals; e.g., product of numbers is
commutative and product of types is commutative up to isomorphism.

\paragraph{Negation: Negation, with intuitionistic and classical logic}
Introduces logical negation as a function into the empty
type, and explains the difference between classical and intuitionistic
logic.

\paragraph{Quantifiers: Universals and existentials}
Recaps universal quantifiers and their correspondence to dependent
functions, and introduces existential quantifiers and their
correspondence to dependent products.

\paragraph{Decidable: Booleans and decision procedures}
Introduces booleans and decidable types, and why the latter is to be
preferred to the former.

\paragraph{Lists: Lists and higher-order functions}
Gives two different definitions of reverse and proves them equivalent.
Introduces map and fold and their properties, including that fold left
and right are equivalent in a monoid.  Introduces predicates that hold
for all or any member of a list, with membership as a specialisation
of the latter.

\subsection*{Part II, Programming Language Foundations}

\paragraph{Lambda: Introduction to lambda calculus}
Introduces lambda calculus, using a representation with named
variables and a separate typing relation. The language used is PCF
\citep{Plotkin-1977}, with variables, lambda abstraction, application,
zero, successor, case over naturals, and fixpoint. Reduction is
call-by-value and restricted to closed terms.

\paragraph{Properties: Progress and preservation}
Proves key properties of simply-typed
lambda calculus, including progress and preservation.  Progress and
preservation are combined to yield an evaluator.

\paragraph{DeBruijn: Inherently typed de Bruijn representation}
Introduces de Bruijn indices and the inherently-typed representation.
Emphasis is put on the structural similarity between a term and its
corresponding type derivation; in particular, de Bruijn indices
correspond to the judgment that a variable is well-typed under a given
environment.

\paragraph{More: More constructs of simply-typed lambda calculus}
Introduces product, sum, unit, and empty types as well as lists and let bindings
are explained.  Typing and reduction rules are given informally; a few
are then give formally, and the rest are left as exercises for the reader.
The inherently typed representation is used.

\paragraph{Bisimulation: Relating reduction systems}
Shows how to translate the language with ``let'' terms
to the language without, representing a let as an application of an abstraction,
and shows how to relate the source and target languages with a bisimulation.

\paragraph{Inference: Bidirectional type inference}
Introduces bidirectional type inference, and applies it to convert from
a representation with named variables and a separate typing relation
to a representation de Bruijn indices with inherent types. Bidirectional
type inference is shown to be both sound and complete.

\paragraph{Untyped: Untyped calculus with full normalisation}
As a variation on earlier themes, discusses an untyped (but inherently
scoped) lambda calculus.  Reduction is call-by-name over open terms,
with full normalisation (including reduction under lambda terms).  Emphasis
is put on the correspondence between the structure of a term and
evidence that it is in normal form.

\subsection*{Discussion}

PLFA and SF differ in several particulars.  PLFA begins with a computationally
complete language, PCF, while SF begins with a minimal language, simply-typed
lambda calculus with booleans.  PLFA does not include type annotations in terms,
and uses bidirectional type inference, while SF has terms with unique types and
uses type checking.  SF also covers a simple imperative language with Hoare logic,
and for lambda calculus covers subtyping, record types, mutable references, and normalisation
---none of which are treated by PLFA.  PLFA covers an inherently-typed de Bruijn
representation, bidirectional type inference, bisimulation, and an
untyped call-by-name language with full normalisation---none of which
are treated by SF.

SF has a third volume, written by Andrew Appel, on Verified Functional
Algorithms. I'm not sufficiently familiar with that volume to have a view
on whether it would be easy or hard to cover that material in Agda.
And SF recently added a fourth volume on random testing of Coq specifications
using QuickChick.  There is currently no tool equivalent to QuickChick
available for Agda.

There is more material that would be desirable to include in PLFA
which was not due to limits of time.  In future years, PLFA may be
extended to cover additional material, including mutable references,
normalisation, System F, pure type systems, and denotational
semantics.  I'd especially like to include pure type systems as they
provide the readers with a formal model close to the dependent types
used in the book.  My attempts so far to formalise pure type systems
have proved challenging, to say the least.

\section{Proofs in Agda and Coq}

\begin{figure}[t]
  \includegraphics[width=\textwidth]{figures/plfa-progress-1.png}
  \caption{PLFA, Progress (1/2)}
  \label{fig:plfa-progress-1}
\end{figure}

\begin{figure}[p]
  \includegraphics[width=\textwidth]{figures/plfa-progress-2.png}
  \includegraphics[width=\textwidth]{figures/plfa-progress-3a.png}
  \caption{PLFA, Progress (2/2)}
  \label{fig:plfa-progress-2}
\end{figure}

\begin{figure}[t]
  \includegraphics[width=\textwidth]{figures/sf-progress-1.png}
  \caption{SF, Progress (1/2)}
  \label{fig:sf-progress-1}
\end{figure}

\begin{figure}[t]
  \includegraphics[width=\textwidth]{figures/sf-progress-2.png}
  \caption{SF, Progress (2/2)}
  \label{fig:sf-progress-2}
\end{figure}

% \begin{figure}[t]
%   \includegraphics[width=\textwidth]{figures/sf-progress-3.png}
%   \caption{SF, Progress (3/3)}
%   \label{fig:sf-progress-3}
% \end{figure}

The introduction listed several reasons for preferring Agda over Coq.
But Coq tactics enable more compact proofs. Would it be possible for
PLFA to cover the same material as SF, or would the proofs
balloon to unmanageable size?

As an experiment, I first rewrote SF's development of simply-typed
lambda calculus (SF, Chapters Stlc and StlcProp) in Agda.  I was a
newbie to Agda, and translating the entire development, sticking as
closely as possible to the development in SF, took me about two days.
I was pleased to discover that the proofs remained about the same
size.

There was also a pleasing surprise regarding the structure of the
proofs. While most proofs in both SF and PLFA are carried out by
induction over the evidence that a term is well typed, in SF the
central proof, that substitution preserves types, is carried out by
induction on terms for a technical reason
(the context is extended by a variable binding, and hence not
sufficiently ``generic'' to work well with Coq's induction tactic). In
Agda, I had no trouble formulating the same proof over evidence that
the term is well typed, and didn't even notice SF's description of the
issue until I was done.

The rest of the book was relatively easy to complete.  The closest to
an issue with proof size arose when proving that reduction is
deterministic.  There are 18 cases, one case per line.  Ten of the
cases deal with the situation where there are potentially two
different reductions; each case is trivially shown to be
impossible.  Five of the ten cases are redundant, as they just involve
switching the order of the arguments.  I had to copy the cases
suitably permuted. It would be preferable to reinvoke the proof on
switched arguments, but this would not pass Agda's termination checker
since swapping the arguments doesn't yield a recursive call on
structurally smaller arguments.  I suspect tactics could cut down the
proof significantly. I tried to compare with SF's proof that reduction
is deterministic, but failed to find that proof.

SF covers an imperative language with Hoare logic, culminating in
code that takes an imperative programme suitably decorated with preconditions
and postconditions and generates the necessary
verification conditions. The conditions are then verified by a custom
tactic, where any questions of arithmetic are resolved by the
``omega'' tactic invoking a decision procedure.  The entire
exercise would be easy to repeat in Agda, save for the last step: I
suspect Agda's automation would not be up to verifying the generated
conditions, requiring tedious proofs by hand.  However, I had already
decided to omit Hoare logic in order to focus on lambda calculus.

To give a flavour of how the texts compare, I show the
proof of progress for simply-typed lambda calculus from both texts.
Figures~\ref{fig:plfa-progress-1} and \ref{fig:plfa-progress-2}
are taken from PLFA, Chapter Properties,
while Figures~\ref{fig:sf-progress-1} and \ref{fig:sf-progress-2}
are taken from SF, Chapter StlcProp.
Both texts are intended to be read online,
and the figures were taken by grabbing bitmaps of the text as
displayed in a browser.

PLFA puts the formal statements first, followed by informal explanation.
PLFA introduces an auxiliary relation \texttt{Progress} to capture
progress; an exercise (not shown) asks the reader to show it isomorphic
to the usual formulation with a disjunction and an existential.
Layout is used to present the auxiliary relation in inference rule form.
In Agda, any line beginning with two dashes is treated as a comment, making
it easy to use a line of dashes to separate hypotheses from conclusion
in inference rules.  The proof of proposition \texttt{progress} (the different
case making it a distinct name) is layed out carefully. The neat
indented structure emphasises the case analysis, and all right-hand
sides line-up in the same column.  My hope as an author is that students
will read the formal proof first, and use it as a tabular guide
to the informal explanation that follows.

SF puts the informal explanation first, followed by the formal proof.
The text hides the formal proof script under an icon;
the figure shows what appears when the icon is expanded.
As a teacher I was aware that
students might skip it on a first reading, and I would have to hope the
students would return to it and step through it with an interactive
tool in order to make it intelligible.  I expect the students skipped over
many such proofs.  This particular proof forms the basis for a question
of the mock exam and the past exams, so I expect most students will actually
look at this one if not all the others.

\newcommand{\ex}{\texttt{x}}
\newcommand{\why}{\texttt{y}}
\newcommand{\EL}{\texttt{L}}
\newcommand{\EM}{\texttt{M}}
\newcommand{\EN}{\texttt{N}}
\newcommand{\tee}{\texttt{t}}
\newcommand{\tick}{\texttt{\lq}}
\newcommand{\GG}{\Gamma}
\newcommand{\AY}{\texttt{A}}
\newcommand{\BE}{\texttt{B}}

(For those wanting more detail: In PLFA, variables and abstractions and
applications in the object language are written $\tick\;\ex$ and
$\lambdabar\;\ex\;{\Rightarrow}\;\EN$ and $\EL\;{\cdot}\;\EM$.  The
corresponding typing rules are referred to by ${\vdash}\tick\;()$
and ${\vdash}\lambdabar\;{\vdash}\EN$ and ${\vdash}\EL\;{\cdot}\;{\vdash}\EM$, where
${\vdash}\EL$, ${\vdash}\EM$, ${\vdash}\EN$ are the proofs that terms
$\EL$, $\EM$, $\EN$ are well typed, and `()` denotes that there cannot
be evidence that a free variable is well typed in the empty context.
It was decided to overload infix dot for readability, but not other
symbols. In Agda, as in Lisp, almost any sequence of characters is a
name, with spaces essential for separation.)

(In SF, variables and abstractions and applications in
the object language are written $\texttt{tvar}~\ex$ and
$\texttt{tabs}~\ex~\tee$ and $\texttt{tapp}~\tee_1~\tee_2$.
The corresponding typing rules are referred to as
$\texttt{T\_Var}$ and $\texttt{T\_Abs}$ and $\texttt{T\_App}$.)

Both Coq and Agda support interactive proof.  Interaction in Coq is
supported by Proof General, based on Emacs, or by CoqIDE, which
provides an interactive development environment of a sort familiar to
most students.  Interaction in Agda is supported by an Emacs mode.

In Coq, interaction consists of stepping through a proof script, at
each point examining the current goal and the variables currently in
scope, and executing a new command in the script.  Tactics are a whole
sublanguage, which must be learned in addition to the language for
expressing specifications.  There are many tactics one can invoke in
the script at each point; one menu in CoqIDE lists about one hundred
tactics one might invoke, some in alphabetic submenus.  A Coq
script presents the specification proved and the tactics executed.
Interaction is recorded in a script, which the students
may step through at their leisure.  SF contains some prose descriptions
of stepping through scripts, but mainly contains scripts that students
are encouraged to step through on their own.

In Agda, interaction consists of writing code with holes, at each
point examining the current goal and the variables in scope, and
typing code or executing an Emacs command.  The number of commands
available is much smaller than with Coq, the most important ones being
to show the type of the hole and the types of the variables in scope;
to check the code; to do a case analysis on a given variable; or to
guess how to fill in the hole with constructors or variables in scope.
An Agda proof consists of typed code.  The interaction is \emph{not}
recorded.  Students may recreate it by commenting out bits of code and
introducing a hole in their place.   PLFA contains some prose descriptions
of interactively building code, but mainly contains code that students
can read.  They may also introduce holes to interact with the code, but
I expect this will be rarer than with SF.

SF encourages students to interact with all the scripts in the text.
Trying to understand a Coq proof script without running it
interactively is a bit like understanding a chess game by reading
through the moves without benefit of a board, keeping it all in your
head.  In contrast, PLFA provides code that students can read.
Understanding the code often requires working out the types, but
(unlike executing a Coq proof script) this is often easy to do in your
head; when it is not easy, students still have the option of
interaction.

While students are keen to interact to create code, I have found they
are reluctant to interact to understand code created by others. For
this reason, I suspect this may make Agda a more suitable vehicle for
teaching.  Nate Foster suggests this hypothesis is ripe to be tested
empirically, perhaps using techniques similar to those of
\citet{Danas-et-al-2017}.

Neat layout of definitions such as that in
Figure~\ref{fig:plfa-progress-2} in Emacs requires a monospaced font
supporting all the necessary characters.  Securing one has proved
tricky. As of this writing, we use FreeMono, but it lacks a few
characters ($\typecolon$ and $\qed$) which are loaded from fonts with a different
width.  Long arrows are necessarily more than a single character wide.
Instead, we compose reduction —→ from an em dash — and an arrow →.
Similarly for reflexive and transitive closure —↠.

\section{Progress + Preservation = Evaluation}

\begin{figure}
  \includegraphics[width=\textwidth]{figures/plfa-eval.png}
  \caption{PLFA, Evaluation}
  \label{fig:plfa-eval}
\end{figure}

A standard approach to type soundness used by many texts,
including SF and PLFA, is to prove progress and preservation,
as first suggested by \citet{Wright-and-Felleisen-1994}.

\begin{theorem}[Progress] Given term $M$ and type $A$ such that
$\emptyset \vdash M : A$ then either $M$ is a value or
$M \longrightarrow N$ for some term $N$.
\end{theorem}

\begin{theorem}[Preservation] Given terms $M$ and $N$ and type $A$
such that $\emptyset \vdash M : A$ and $M \longrightarrow N$, then
$\emptyset \vdash N : A$.  \end{theorem}

A consequence is that when a term reduces to a value it retains
the same type.  Further, well-typed terms don't get stuck:
that is, unable to reduce further but not yet reduced to a value.
The formulation neatly accommodates the case of non-terminating
reductions that never reach a value.

One useful by-product of the formal specification of a programming
language may be a prototype implementation of that language.  For
instance, given a language specified by a reduction relation, such
as lambda calculus, the prototype might accept a term and apply reductions
to reduce it to a value.  Typically, one might go to some extra work to
create such a prototype.  For instance, SF introduces a \texttt{normalize}
tactic for this purpose.  Some formal methods frameworks, such as
Redex \citep{Felleisen-et-al-2009} and K \citep{Rosu-Serbanuta-2010},
advertise as one of their advantages that they can generate
a prototype from descriptions of the reduction rules.

I was therefore surprised to realise that any constructive proof of
progress and preservation \emph{automatically} gives rise to such a
prototype.  The input is a term together with evidence the term is
well-typed.  (In the inherently-typed case, these are the same thing.)
Progress determines whether we are done, or should take another step;
preservation provides evidence that the new term is well-typed, so we
may iterate. In a language with guaranteed termination, we cannot
iterate forever, but there are a number of well-known techniques to
address that issue; see, e.g., \citet{Bove-and-Capretta-2001},
\citet{Capretta-2005}, or \citet{McBride-2015}.
We use the simplest, similar to McBride's \emph{petrol-driven} (or
\emph{step-indexed}) semantics: provide a maximum number of steps to
execute; if that number proves insufficient, the evaluator returns the
term it reached, and one can resume execution by providing a new
number.

Such an evaluator from PLFA is shown in Figure~\ref{fig:plfa-eval},
where (inspired by cryptocurrencies) the number of steps
to execute is referred to as \emph{gas}. All of the example reduction
sequences in PLFA were computed by the evaluator and then edited to
improve readability; in addition, the text includes examples of
running the evaluator with its unedited output.

It is immediately obvious that progress and preservation make it
trivial to construct a prototype evaluator, and yet I cannot find such
an observation in the literature nor mentioned in an introductory
text.  It does not appear in SF, nor in \citet{Harper-2016}.  A plea
to the Agda mailing list failed to turn up any prior mentions.
The closest related observation I have seen in the published
literature is that evaluators can be extracted from proofs of
normalisation \citep{Berger-1993,Dagand-and-Scherer-2015}.

(Late addition: My plea to the Agda list eventually bore fruit.  Oleg
Kiselyov directed me to unpublished remarks on his web page where he
uses the name \texttt{eval} for a proof of progress and notes ``the
very proof of type soundness can be used to evaluate sample
expressions'' \citep{Kiselyov-2009}.)


\section{Inherent typing is golden}

\begin{figure}[p]
  \includegraphics[width=\textwidth]{figures/raw.png}
  \caption{Raw approach in PLFA}
  \label{fig:raw}
\end{figure}

\begin{figure}[t]
  \includegraphics[width=\textwidth]{figures/inherent.png}
  \caption{Inherent approach in PLFA}
  \label{fig:inherent}
\end{figure}

The second part of PLFA first discusses two different approaches to
modelling simply-typed lambda calculus.  It first presents raw
terms with named variables and a separate typing relation and
then shifts to inherently-typed terms with de Bruijn indices.
Before writing the text, I had thought the two approaches
complementary, with no clear winner.  Now I am convinced that the
inherently-typed approach is superior.

Figure~\ref{fig:raw} presents the raw approach.
It first defines $\texttt{Id}$, $\texttt{Term}$,
$\texttt{Type}$, and $\texttt{Context}$, the abstract syntax
of identifiers, raw terms, types, and contexts.
It then defines two judgements,
$\GG\:{\ni}\:\ex\:{\typecolon}\:\AY$ and
$\GG\:{\vdash}\:\EM\:{\typecolon}\:\AY$,
which hold when under context $\GG$ the variable $\ex$
and the term $\EM$ have type $\AY$, respectively.

Figure~\ref{fig:inherent} presents the inherent approach.
It first defines $\texttt{Type}$ and $\texttt{Context}$, the abstract syntax
of types and contexts, of which the first is as before and the second is
as before with identifiers dropped.  In place of the two judgements,
the types of variables and terms are indexed by a context and a type,
so that $\GG\:{\ni}\:\AY$ and $\GG\:{\vdash}\:\AY$ denote
variables and terms, respectively, that under context $\GG$ hae type $\AY$.
The indexed types closely resemble the previous judgements:
we now represent a variable or a term by the proof that it is well-typed.
In particular, the proof that a variable is well-typed in the raw approach
corresponds to a de Bruijn index in the inherent approach.

The raw approach requires more lines of code than
the inherent approach.  The separate definition of raw
terms is not needed in the inherent
approach; and one judgement in the raw approach needs to check that
$\ex \not\equiv \why$, while the corresponding judgement in the
inherent approach does not.  The difference becomes more pronounced
when including the code for substitution, reductions, and proofs of
progress and preservation.  In particular, where the raw approach
requires one first define substitution and reduction and then prove
they preserve types, the inherent approach establishes substitution at
the same time it defines substitution and reduction.

Stripping
out examples and any proofs that appear in one but not the other (but
could have appeared in both), the full development in PLFA for the raw approach takes 451
lines (216 lines of definitions and 235 lines for the proofs) and the
development for the inherent approach takes 275 lines (with
definitions and proofs interleaved).  We have 451 / 235 = 1.64,
close to the golden ratio.

The inherent approach also has more expressive power.  The raw
approach is restricted to substitution of one variable by a closed
term, while the inherent approach supports simultaneous substitution
of all variables by open terms, using a pleasing formulation due to
\citet{McBride-2005}, inspired by \citet{Goguen-and-McKinna-1997} and
\citet{Altenkirch-and-Reus-1999} and described in
\citet{Allais-et-al-2017}. In fact, I did manage to write a variant of
the raw approach with simultaneous open substitution along the lines
of McBride, but the result was too complex for use in an introductory
text, requiring 695 lines of code---more than the total for the other
two approaches combined.

The text develops both approaches because the raw approach is more
familiar, and because placing the inherent approach first would
lead to a steep learning curve.  By presenting the more long-winded
but less powerful approach first, students can see for themselves the
advantages of de Bruijn indices and inherent types.

There are actually four possible designs, as the choice of named
variables vs de Bruijn indices, and the choice of raw vs
inherently-typed terms may be made independently.  There are synergies
beween the two.  Manipulation of de Bruijn indices can be notoriously
error-prone without inherent-typing to give assurance of correctness.
In inherent typing with named variables, simultaneous substitution by
open terms remains difficult.

The benefits of the inherent approach
are well known to some. The technique was introduced by
\citet{Altenkirch-and-Reus-1999}, and widely used elsewhere,
notably by \citet{Chapman-2009} and \citet{Allais-et-al-2017}.
I'm grateful to David Darais for bringing it to my attention.


\section{Conclusion}

I look forward to experience teaching from the new text,
and encourage others to use it too.  Please comment!


\paragraph{Acknowledgements}

A special thank you to my coauthor, Wen Kokke.  For inventing ideas on
which PLFA is based, and for hand-holding, many thanks to Conor
McBride, James McKinna, Ulf Norell, and Andreas Abel.  For showing me
how much more compact it is to avoid raw terms, thanks to David
Darais.  For inspiring my work by writing SF, thanks to Benjamin
Pierce and his coauthors.  For comments on a draft of this paper, an
extra thank you to James McKinna, Ulf Norell, Andreas Abel, and
Benjamin Pierce. This research was supported by EPSRC Programme Grant
EP/K034413/1.


\bibliographystyle{plainnat}
%\bibliographystyle{splncsnat.bst}
\bibliography{PLFA}



\end{document}
