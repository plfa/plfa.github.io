%%%%%%%%%%%%%%%%%%%%%%%%%%%%%%%%%%%%%%%%%%%%%%%%%%%%%%%%%%%%%
% Instructions for TSPL Exam
% %%%%%%%%%%%%%%%%%%%%%%%%%%%%%%%%%%%%%%%%%%%%%%%%%%%%%%%%%%%%%

\documentclass[12pt]{article}
\usepackage{a4,amssymb}
% \setlength{\oddsidemargin}{-1.5cm}
% \addtolength{\textwidth}{2cm}
% \addtolength{\textheight}{3cm}

\begin{document}
\pagestyle{empty}
\setcounter{page}{1}

\begin{center}
\large Types and Semantics for Programming Languages
\end{center}

\section*{Instructions}

\begin{enumerate}

\item
  The exam lasts two hours.

\item
  Place your student identity card face-up on the desk in front of you.  The
  invigilator may come to check your identity, and in this case you may be asked
  to allow the invigilator to briefly use your computer.  The exam time has been
  calculated to allow time for such interruptions.

\item
  You may log into your computer as soon as you are ready to do so.

\item
  To download the exam paper, open a terminal window and type:
  \begin{center}
    \texttt{getpapers}
  \end{center}
  This will create a subdirectory \texttt{tspl-mpe} in your home directory,
  containing the following files.
  \begin{center}
    \begin{tabular}{ll}
      \texttt{tspl-mpe/papers/exam.pdf}                & the exam \\
      \texttt{tspl-mpe/papers/instructions.pdf}        & these instructions \\
      \texttt{tspl-mpe/templates/Exam.lagda}           & exam template to edit \\
      \texttt{tspl-mpe/original\_templates/Exam.lagda} & backup of exam template
    \end{tabular}
  \end{center}
  The directories \texttt{tspl-mpe/papers}, \texttt{tspl-mpe/original\_templates}
  are read only, but you may read and write \texttt{tspl-mpe/templates}.

\item
  To setup the Agda environment, open a terminal window and type:
  and type:
  \begin{center}
    \texttt{tspl-setup}
  \end{center}
  This will also open a browser, with three tabs which contain the textbook
  \emph{Programming Language Foundations in Agda}, the documentation for the
  Agda standard library, and the documentation for Agda are available.

{\it (Note that internet access has been disabled.)}

\begin{center}
  \textbf{Do nothing further until the start of the exam is announced!}
\end{center}

\hfill \textit{Please Turn Over}
\newpage

\item When the start of the exam is announced, open the exam paper
\begin{center}
  \texttt{tspl-mpe/papers/exam.pdf}
\end{center}
with the standard PDF viewer.

\item Change to the template directory
\begin{center}
  \texttt{cd tspl-mpe/templates}
\end{center}
and edit the file
\begin{center}
  \texttt{Exam.lagda}
\end{center}
to include your answers, using Emacs or anything else suitable.
You are recommended to save your work on a regular basis.

\item Before submitting, make sure your file is processed by Agda
  correctly. Code which prevents the file from compiling should be
  made into comments. If you fail to solve part of a problem, you
  may get more credit if you indicate clearly which part you have
  not solved. 

\item \emph{Please ensure before submission that the file}
  \texttt{Exam.lagda} \emph{contains your solutions to the exam.}  Submit
  your file using the command:
  \begin{center}
  \texttt{examsubmit Exam.lagda}
  \end{center}
  If you get an error, please check carefully that your file is called
  \texttt{Exam.lagda} and that you are in the same directory as this
  file. If you continue to have problems, please contact one of the
  invigilators.
 
Repeated submit commands are allowed, and will overwrite previous
submissions.  The last file submitted will be the one marked.

\item When the invigilators announce the end of the exam, you must
  submit and log out immediately.

\end{enumerate}

\end{document}


%%% Local Variables: 
%%% mode: latex
%%% TeX-master: t
%%% End: 
